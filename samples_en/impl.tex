\chapter{Developer documentation}
\label{ch:impl}

This chapter gives the very details of involved technologies, development environment, analysis, implementations and considerations.

\section{Environment Requirements}
The application is supported to be ran and developed both locally and with dedicated cloud application: 
\begin{enumerate}
    \item Possible toolkit for Local environment
        \begin{itemize}
            \item Eclipse with Spring Tools and CAP extensions
            \item Visual Studio Code with SAP CAP extensions
        \end{itemize}
    \item Possible toolkit for Cloud environment
        \begin{itemize}
            \item SAP Business Application Studio with Full-Stack Development dev-space (recommanded, used in this thesis)
        \end{itemize}
\end{enumerate}

For local development, up to date Java JDK, node.js and CAP extensions are hard requirements. The specific set up can be find \hyperlink{https://developers.sap.com/tutorials/btp-app-prepare-dev-environment-cap.html}{here}.

This thesis used SAP Business Application Studio (BAS) as development tool, which is ready for use. 

A snapshot of versions at the time of development:

\lstset{caption={Version check}, label=src:bash}
\begin{lstlisting}[language={bash}]
> java --version
# openjdk 17.0.4.1 2022-08-12 LTS
# OpenJDK Runtime Environment SapMachine (build 17.0.4.1+1-LTS)
# OpenJDK 64-Bit Server VM SapMachine (build 17.0.4.1+1-LTS, mixed mode, sharing)
> cds version
# @sap/cds: 7.2.1
# @sap/cds-compiler: 4.0.2
# @sap/cds-dk: 7.2.0
# @sap/cds-dk (global): 7.2.0
# @sap/cds-fiori: 1.1.0
# @sap/cds-foss: 4.0.2
# @sap/cds-mtxs: 1.11.0
# @sap/eslint-plugin-cds: 2.6.3
# Node.js: v18.14.2

\end{lstlisting}


\section{Run}
The following commands can be used to run the application for testing.

\lstset{caption={Run commands}, label=src:bash}
\begin{lstlisting}[language={bash}]
cds watch # This ... explain!!
\end{lstlisting}
\begin{lstlisting}[language={bash}]
mvn clean install
mvn spring-boot:run
\end{lstlisting}


\section{Analysis and Design}
\subsection{User Story}
\subsection{User Cases}
\subsection{User Diagrams}

\begin{figure}[H]
	\centering
	\includegraphics[height=250px]{images/User_Diagram-StorageService.png}
	\caption{Storage Service}
	\label{fig:service-1}
\end{figure}
\section{Application Structure}
\subsection{Overview}
\subsection{Project Structure}
\section{Data Model}
Since Digital Lab provides more then one solutions and to ensure the maximum reuse of models, data model is seperated into 2 parts: solution specific data models and common data models.
\subsection{Specification}
\subsubsection{Solution specific data model}
picture
\subsubsection{General common data model}
picture
\subsection{Implementation}
All data models are modeled by CDS' definition language. Single entities are put into their own \textit{EntityName.cds} files. The main goal is to model out the exact relationships as illustrate in the specification section.

Here list the structure of folders:
\lstset{caption={Structure of folders - data model}, label=src:bash}
\begin{lstlisting}[language={bash}]
root/
    |- db/
        |- data
        |- model
            |- packagehandling 
                |- *.cds # Definition of solution specific data model
            |- common
                |- *.cds # Definition of common data model
\end{lstlisting}

The solution-specific and common data models are also seperated in the sense of names spaces by define the following accordingly at the top of the entities definitions.

\lstset{caption={cds namespaces for data model}, label=src:cds}
\begin{lstlisting}[language={c++}]
namespace com.sap.internal.digitallab.packagehandling.common; // namespace for reusful 
namespace com.sap.internal.digitallab.packagehandling.core; // namespace for solution-specific
\end{lstlisting}

To optimize and standardize the outcome, common \textbf{aspects} (\textit{cuid, managed, User, CodeList}) from \textit{@sap/cds/common} are used in the definition. It can be seen as "extend" in the object-orient way, or more speificly "interface" in Java. Aspects are defined in a similiar way as entities, consists of fields/properties. Entities can "implement" zero to more aspects, "inheriting" aspect's properties and "extend" with its own properties. 

\begin{definition}
    \textbf{CDS's aspects} allow to flexibly extend definitions by new elements as well as overriding properties and annotations. They're based on a mixin approach as known from Aspect-oriented Programming methods.
\end{definition}

Here records the common aspects used by this thesis. \textit{managed} comes with 4 elements: created, created, changed, changed, and automatically updates them. \textit{cuid} automatically assigns UUID primary key to entity. \textit{CodeList} provides a out of box way of implementing enumeration like data structures and supports automatic localization.

\lstset{caption={Aspect managed from @sap/cds/common}, label=src:bash}
\begin{lstlisting}[language={c++}]
aspect managed {
  createdAt  : Timestamp @cds.on.insert : $now;
  createdBy  : User      @cds.on.insert : $user;
  modifiedAt : Timestamp @cds.on.insert : $now  @cds.on.update : $now;
  modifiedBy : User      @cds.on.insert : $user @cds.on.update : $user;
}
\end{lstlisting}

\lstset{caption={Aspect cuid from @sap/cds/common}, label=src:bash}
\begin{lstlisting}[language={c++}]
aspect cuid {
  key ID : UUID; //> automatically filled in
}
\end{lstlisting}

\lstset{caption={Aspect CodeList from @sap/cds/common}, label=src:bash}
\begin{lstlisting}[language={c++}]
aspect CodeList @(
    cds.autoexpose,
    cds.persistence.skip : 'if-unused'
) {
    name  : localized String(255)  @title : '{i18n>Name}';
    descr : localized String(1000) @title : '{i18n>Description}';
}
\end{lstlisting}

Below captures two examples from the thesis on usage of aspects coupled with the SQL DDL corresponding to the cds definition.
\lstset{caption={managed, cuid delivery company entity and corresponding SQL DDL}, label=src:sql}
\begin{lstlisting}[language={sql}]
entity DeliveryCompany : cuid, managed {
    name     : String(255) not null;
    logo     : String(255)  @Core.IsURL  @Core.IsMediaType;
    packages : Association to many Package
                   on packages.deliveryCompany = $self;
}

CREATE TABLE com_sap_internal_digitallab_packagehandling_core_DeliveryCompany (
  ID NVARCHAR(36) NOT NULL,
  createdAt TIMESTAMP_TEXT,
  createdBy NVARCHAR(255),
  modifiedAt TIMESTAMP_TEXT,
  modifiedBy NVARCHAR(255),
  name NVARCHAR(255) NOT NULL,
  logo NVARCHAR(255),
  PRIMARY KEY(ID)
);
\end{lstlisting}

\lstset{caption={codelist package type entity and corresponding SQL DDL}, label=src:sql}
\begin{lstlisting}[language={sql}]
entity PackageType : sap.common.CodeList {
    key code : String(255) not null;
}

CREATE TABLE com_sap_internal_digitallab_packagehandling_core_PackageType (
  name NVARCHAR(255),
  descr NVARCHAR(1000),
  code NVARCHAR(255) NOT NULL,
  PRIMARY KEY(code)
);
\end{lstlisting}

\section{UI}
\subsection{Specification}
\subsection{Implementation}

\section{Services}
\subsection{Specification}
\subsection{Implementation}

\section{Testing}
\section{Deployment}

\section{Data Model}

\section{Technologies and Terms}
