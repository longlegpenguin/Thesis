\chapter{Introduction}
\label{ch:intro}

Alongside the rapid growth of e-commerce and delivery capability, waiting for delivery has became an unavoidable situation for a person. In most of the cases, parcels arrive during the working hours, and it is ineffective and not always possible for someone to stay for the delivery. Thereby, SAP offers its employees with a solution, parcel collection service. Employees could deliver their personal deliveries to the address of the company. By the time it arrives, it is first accepted by the 24/7 reception desk and is able to be picked by the employee any time later.

Digital Lab is a team aiming to provide internal applications to improve employees' life at SAP. Throughout the years, dozens of solutions were created and deployed to SAP BTP Neo platform. With the deprecation of Neo, the old solutions will soon be forced to retire due to their dependence on the platform, including the parcel collection service.

This thesis creates a brand-new parcel collection application to fit in the gap. The application is designed to be platform independent and to optimize the using experience of every targeted groups of users (administrator, manager, receptionist and end-user). It provides a simplified work stream that is, receptionist register a new package, receptionist confirms a new package by allocating it a slot, end-user receives email and confirms the pick up. It also provides the authorized party actions to manage the storage, parcels and delivery companies. Own histories can be reviewed by registered users.

Technically, the application is implemented as a MTA following CAP programming model and runs on SAP BTP, Cloud Foundry environment. It consists of 7 micro services. The data model and UI of the application is modeled and annotated with CDS language, which is the backbone of CAP. The back-end services are also defined in CDS and custom logic are implemented with the help of Java Spring Boot framework. 

The more specific usages and implementations can be checked later in Chapter 2 and Chapter 3.

 velit \cite{dahl1972structured}. euismod.\footnote{Maecenas a urna viverra, scelerisque nibh ut, malesuada ex.}

 mus \cite{cormen2009algorithms,krasner1988mvc}. bibendum  \citeauthor{dijkstra1979goto}  purus \cite{dijkstra1979goto}. 
