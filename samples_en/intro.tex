\chapter{Introduction}
\label{ch:intro}

\section{Motivation}
Alongside the rapid growth of e-commerce and delivery capability, waiting for delivery has became an unavoidable situation for a person. In most of the cases, parcels arrive during the working hours, and it is ineffective and not always possible for someone to stay for the delivery. Thereby, SAP \cite{sap} offers its employees with a solution, parcel collection service. Employees could deliver their personal deliveries to the address of the company. By the time it arrives, it is first accepted by the 24/7 reception desk and is able to be picked by the employee any time later.

Digital Lab is a team aiming to provide internal applications to improve employees' life at SAP. Throughout the years, dozens of solutions were created and deployed to SAP BTP \cite{btp} Neo platform \cite{neo}. With the deprecation of Neo, the old solutions will soon be forced to retire due to their dependence on the platform, including the parcel collection service.

This thesis creates a brand-new parcel collection application to fit in the gap. The application is designed to be platform independent and to optimize the using experience of every targeted groups of users (administrator, manager, receptionist and end-user). It provides a simplified work stream that is, receptionist register a new package, receptionist confirms a new package by allocating it a slot, end-user receives email and confirms the pick up. It also provides the authorized party actions to manage the storage, parcels and delivery companies. Own histories can be reviewed by registered users.

The resulting solution consists of 6 interdependent applications consuming different services sharing the same database. The related information regarding the solution are collected in this thesis. The coming \autoref{sec:IntroSum} gives a quick preview of the upcoming chapters.

% Technically, the application is implemented as a MTA following CAP programming model and runs on SAP BTP, Cloud Foundry environment. It consists of 7 micro services. The data model and UI of the application is modeled and annotated with CDS language, which is the backbone of CAP. The back-end services are also defined in CDS and custom logic are implemented with the help of Java Spring Boot framework. 

\section{Summary}
\label{sec:IntroSum}

In the forthcoming chapters, a detailed exploration of the parcel collection application is compiled and presented. 

The \autoref{ch:user} contains the user documentation, in which the purpose and usage of application is explained. The chapter falls into two parts. In the first part, general prerequisites and related role concepts of the application are explained. The second part focusing on the practical aspects tailored for end-users, receptionists, facility managers, and administrators. Role by role, the usages of each application are explained step by step, page after page embedded with rich images illustrations, providing any potential users with a clear idea of the goal of the application and the things can be done with. 

After this comes \autoref{ch:impl}, the developer documentation. This chapter reveals all critical technical details lies under the application. It illustrates the general application structure and logs the required local environments for potential future developments. It explains in details the architectures and implementations of front-end, back-end and data model. It also examines the testing ideas and procedures. 

Finally a conclusion (\autoref{ch:sum}) is attached, re-summarizing the solution and listing out the main take ways from the solution. It also concludes the potential improvements from a developer's perpect of view.

As readers progress through the subsequent chapters, one will gradually get to know the application inside-out, from both user and developer perspectives. One can not only learn how to make use of the solution, but also how to further develop the application.
Here jumps to the user documentation (\autoref{ch:user}).


 % velit \cite{dahl1972structured}. euismod.\footnote{Maecenas a urna viverra, scelerisque nibh ut, malesuada ex.}

 % mus \cite{cormen2009algorithms,krasner1988mvc}. bibendum  \citeauthor{dijkstra1979goto}  purus \cite{dijkstra1979goto}. 
