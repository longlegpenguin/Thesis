\chapter*{Conclusion}
\label{ch:sum}

% This \autoref{ch:sum} brings us to the end of our paper. 

This chapter quickly walks through how the original motivation had been fulfilled by the project. The side advantages and the future considerations of the project are also stated. Finally, the main personal takeaways concludes the entire thesis.

%\subsection{Achievement of Objectives}
By the end of the thesis, a working prototype of the new parcel collection service application has been established.
As being stated in \autoref{ch:intro}, the most significant catalyst for the establishment of this thesis was the deprecation of the Neo platform, which leads to the abandon of the old parcel handling software. This problem is resolved at this point, as the new solution is constructed over a platform independent framework and could be deployed to the currently needed Cloud Foundry environment \cite{cf}. From the usage aspect, the solution of this thesis provides a much richer (capability-wise) yet simplified (process-wise) functionalities than the present application. In this sense, the thesis had accomplished the primary objectives.

%\subsection{Other Benefits}
Apart from simply a new solution, this thesis can also serve as an up to date example for other Digital Lab projects. It is a concise living show case of the combination of the evolving CAP framework Java flavour, Fiori elements and UI5 framework, which all are today's latest SAP development toolkit. 

%\subsection{Future Consideration}
The nearest future step could be the trial deployment. At this stage of development, it could benefit a lot from the CD/CI \cite{cdci} pipelines, so any new adaptions can be release and tests continuously. Also more decorations and routing between applications can be added to the UI, adapting more features of Fiori elements.

%\subsection{Concluding Remarks}
In short, the development process, though challenging, has been meaningful for me. I gained a much deeper understanding of the software development cycles and also get valuable skills on the SAP development. It is a very nice showcase of the knowledge I have accumulated in the past three years and it extends beyond.
